\documentclass[12pt]{article}

% Margins
\usepackage[letterpaper, top=1in, bottom=1in, left=.5in, right=.5in]{geometry}

% For prettier tables
\usepackage{array}


% For code
\usepackage{courier}
\usepackage{listings}
\lstset{ mathescape }
\lstset{basicstyle=\ttfamily\footnotesize,breaklines=true}

\usepackage{hyperref}

% Change Font
\usepackage[sfdefault]{roboto}  %% Option 'sfdefault' only if the base font of the document is to be sans serif
\usepackage[T1]{fontenc}

% For double spacing
\usepackage{setspace}

\usepackage[english]{babel}
\usepackage[utf8]{inputenc}
\usepackage{fancyhdr}

\pagenumbering{arabic}

\pagestyle{fancy}
\rhead{Jimmy Hickey}

\lhead{Research Proposal}


%opening
\title{
Computer Corrected Color Blindness: Proposal
}
\author{Jimmy Hickey}



\begin{document}
\maketitle
\doublespacing


\section{Introduction}

Color blindness is a deficiency in a person's color vision. A color blind person will often confuse different colors that a person with normal vision would not have trouble distinguishing. The most common type of color blindness is red-green, followed by blue-yellow, and the rare full lack of color vision. Red-green color blindness effects "as many as 8 percent of men and 0.5 percent of women with Northern European ancestry." (National Eye Institute) The effects of color blindness can be as range from minor disruptions such as choosing a mismatched outfit to more major difficulties such as discern between the colors on traffic lights.
There is no cure for color blindness, however there exist some methods which those affected can alleviate their symptoms. 

One popular solution to color blindness is specialized glasses. The company enchroma sells a variety of these glasses. Color vision starts in the cones of the eye, of which people have three varieties: green, red, and blue. These detect and distinguish between different wavelengths of light; however, in the eye of a color blind person, this discrimination is not always made properly. The enchroma glasses filter out the overlapping wavelengths, making distinctions (particularly between red and green) clearer. These glasses range from \$349 to \$429 without prescriptions and are not ``intended to help pass color blindness tests for occupational purposes." (enchroma) Additionally, they (enchroma brand) are not useful to people with blue-yellow color blindness. A more permanent solution is currently being tested as well.

There is experimental gene therapy that has produced promising results in animals. In one study squirrel monkeys, which are naturally red-green color blind, were made able to perceive differences in the two colors after treatment. These animals were monitored and ``retained their new tricolor sensory capacity for more than two years." (MIT Technology Review). Additionally, there have been no detected harmful side effects from the treatment; four more animals have been successfully treated since. These auspicious trials suggest that there may be a permanent fix to color blindness in the near future; however, gene therapy is often prohibitively expensive, costings hundreds of thousands of dollars per patient. A short-term, cheap solution to some of the issues faced by the color blind is needed. 

The burgeoning field of computer vision lends itself perfectly to this problem. Much work has been done in the area of feature recognition. Projects like Google's deep dream and work in robotic vision are some of the wide applications of computer vision. Computers' abilities to analyze images is greatly enhancing. These methods can be used as the small, cheap fix to some color blind issues. A first step would be to examine images.

Instead of recognizing complex features, a machine needs only to examine the colors in an image. With appropriate knowledge of the problem domain, it should be able to diagnose whether there is areas of the image that may cause issues for the color blind. For example, given a set of color blind tests, it would be able to tell which slides a color blind person would have trouble with. This process can then be incrementally furthered.

\begin{enumerate}
	\item The system can mark all problem areas.
	\item The system can fix the problem areas without losing too much of the context of the image.
	\item The system can perform the preceding actions on a video.
	\item The system can perform the preceding actions on a live video feed.
\end{enumerate}

For the scope of this project, the first and second object will be addressed.

Unlike the glasses, the machine is not dealing with the light itself, but how it is displayed. Thus, this could be easily expanded to support all of the common types of color blindness. 


\section{Hypothesis}
A system can be trained to identify and resolve areas of potential color blind confusion in images.


\section{Methods}

 A data set will be created using Ishihara Color Tests. A set of color blind people can determine which slides they have problems reading. These will be given to a neural network, where the system will learn the patterns behind these issues.
 Once the Ishihara data set has been exhausted the system will have learned enough to be trained a wider breadth of images. 
 
 Once the system can confidently identify color blind problem areas, it will be taught to correct them. 

\section{References}
\singlespacing
\begin{enumerate}
	\item 
		\href{https://nei.nih.gov/health/color_blindness/facts_about}{National Eye Institute}

	\item
		\href{https://www.technologyreview.com/s/601782/how-enchromas-glasses-correct-color-blindness/}{MIT Technology Review of Enchroma glasses}
	
	\item
		\href{http://enchroma.com/contact-us/}{Enchroma website}
		
	\item
		\href{https://www.technologyreview.com/s/415339/color-blind-monkeys-get-full-color-vision/}{MIT Technology Review of Color Blind Monkey Gene Therapy}
		
	\item
		\href{https://arxiv.org/abs/1711.10662}{An Adaptive Fuzzy-Based System to Simulate, Quantify and Compensate Color Blindness}
		
	\item
		\href{https://pdfs.semanticscholar.org/e9e7/d61430c6438fc52031111e48f1224e14fa5a.pdf}{COLORIZATION OF GRAYSCALE IMAGES: AN OVERVIEW}

\end{enumerate}

\end{document}
