\documentclass[12pt]{article}

% Margins
\usepackage[letterpaper, top=1in, bottom=1in, left=.5in, right=.5in]{geometry}

% For prettier tables
\usepackage{array}


% For code
\usepackage{courier}
\usepackage{listings}
\lstset{ mathescape }
\lstset{basicstyle=\ttfamily\footnotesize,breaklines=true}

\usepackage{hyperref}

% Change Font
\usepackage[sfdefault]{roboto}  %% Option 'sfdefault' only if the base font of the document is to be sans serif
\usepackage[T1]{fontenc}

% For double spacing
\usepackage{setspace}

\usepackage[english]{babel}
\usepackage[utf8]{inputenc}
\usepackage{fancyhdr}

\pagenumbering{arabic}

\pagestyle{fancy}
\rhead{Jimmy Hickey}

\lhead{Research Proposal}


%opening
\title{
Computer Corrected Color Blindness: Proposal
}
\author{Jimmy Hickey}



\begin{document}
\maketitle
\doublespacing


\section{Introduction}

One popular solution to color blindness is wearing specialized glasses.
There is experimental gene therapy that has produced promising results in animals. 

\section{Hypothesis}
Using machine learning algorithms, a computer can be trained to identify and resolve areas of potential color blind confusion in images.

\section{Methods}

\section{References}
\singlespacing
\begin{enumerate}
	\item 
		\href{https://nei.nih.gov/health/color_blindness/facts_about}{National Eye Institute}

	\item
		\href{https://www.technologyreview.com/s/601782/how-enchromas-glasses-correct-color-blindness/}{MIT Technology Review of Enchroma glasses}
	\item
		\href{http://www.newsweek.com/2015/05/22/cure-color-blindness-isnt-just-monkey-business-330258.html}{Newsweek article on gene therapy}
\end{enumerate}

\end{document}
